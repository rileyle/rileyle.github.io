\documentclass[12pt]{article}

\usepackage{graphics}
\usepackage{html}
\usepackage{amssymb}
\usepackage{hyperref}

\renewcommand{\familydefault}{\sfdefault}

\setlength{\textwidth} {6.5 true in}
\setlength{\textheight}{9 true in}
\setlength{\hoffset}   {-0.50 true in}
\setlength{\voffset}   {-0.75 true in}

\begin{document}

\thispagestyle{empty}

\begin{latexonly}
  \section*{Uncertainties}
\end{latexonly}

\subsection*{Random or Statistical Uncertainties}

Random or statistical uncertainties correspond to random variations in
the results of repeated measurements. These random variations are
sometimes due to limitations of the measuring device. For example,
electronic noise and air currents lead to random fluctuation in motion
detector readings. These fluctuations occur even when the motion
detector is measuring the distance to a stationary object. Random
variations in repeated measurements can also be a characteristic of
the system being measured. For example, if we use a meter stick to
measure the landing positions of a series of projectiles shot from a
spring-loaded launcher, we see significant random variations which
clearly do not arise from the limitations of the meter stick. Instead,
we suspect that the launch velocity given to projectiles by the
launcher is subject to small random variations. 

Truly random variations average to zero, and so the way to remove
them is to average several measurements,
\begin{equation}
\bar{x} = \frac{1}{N} \sum_{i=1}^N x_i
\end{equation}
The \textbf{average value}, or \textbf{mean value}, $\bar{x}$
approaches the ``true value'' as the number of measurements in the
average approaches infinity. Finding the ``true value'' is
impractical, so we must settle for the ``best value'' given by the
average of a finite set of measurements. When working with more than
just a few measurements, it is a good idea to use the
\texttt{average()} function of your spreadsheet (or the 
\texttt{numpy.average()} function if you are using Python).

Random variations are described by the normal distribution, or
Gaussian distribution, or ``bell curve.''  The uncertainty in the
``best value'' of a large collection of normally distributed
measurements can be calculated using the \textbf{standard deviation}
\begin{equation} 
\sigma_x = \sqrt{\frac{1}{N-1} \sum_{i=1}^N (x_i - \bar{x})^2} 
\end{equation}
which describes the width of the distribution. More precisely, about
68\% of a normal distribution falls within $\pm \sigma$ of the average
value. The standard deviation is the uncertainty in a single
measurement in the distribution. Rather than doing this calculation by
hand, use the \texttt{stdev()} function of your spreadsheet (or the 
\texttt{numpy.std()} function if you are using Python).

The uncertainty in the average of a collection of measurements is less
than $\sigma$. This follows from the idea that the more measurements
we make, the closer the average value comes to the ``true value.'' The
\textbf{standard deviation of the mean} is given by 
\begin{equation}
\sigma_{\bar{x}} = \frac{\sigma}{\sqrt{N}} 
\end{equation}
We report this as the uncertainty in $\bar{x}$. (The \texttt{sqrt()} function
of your spreadsheet or \texttt{numpy} computes square roots.)

\subsection*{Systematic Errors}

Systematic errors are are due to a defect in the equipment or methods
used to make measurements. For example, a motion sensor can be poorly
calibrated so that it gives distance readings which are only 90\% of
the true values. It has a systematic uncertainty (10\%) that is much
greater in magnitude than the statistical uncertainty in its
readings. Systematic errors are often difficult to detect, because
they do not show up as variations in the results of repeated
measurements. It is important to think about possible sources of
systematic errors and to try to correct them or rule them out, for
example by 
\begin{itemize}
\item checking calibrations
\item comparing results obtained using independent means (comparing
  measurements with meter sticks and motion detectors)
\item comparing results with accepted values (when available)
\end{itemize}

\subsection*{Estimating Uncertainties}

We often do not have the luxury of a large collection of normally
distributed measurements to analyze. Instead, we must somehow estimate
the uncertainty of a single measurement. This is necessarily somewhat
subjective. If only a few measurements are available, it is more
reasonable to use the entire range covered by the measurements to
define the uncertainty instead of calculating the standard deviation
of the mean. If only one measurement is available the resolution of
the device and the variation in the quantity measured are important
guides. For example, the resolution of a meter stick is 1~mm. If it is
used to measure the length of a rectangular steel plate, an
uncertainty of 1~mm, or perhaps even 0.5~mm, is reasonable. If I use
the same meter stick to measure the height of a small child, issues
of variable posture and how I line up the stick lead to an uncertainty
of as much as 1~cm. Be conservative with your estimates. That is, when
in doubt, it is a good policy to report a larger uncertainty.

\subsection*{Propagation of Uncertainties in Calculations}

Frequently, calculations involve one or more measured quantities, and we
need to determine how the uncertainties in input quantities translate
into the uncertainty in the result. The guidelines below cover all
of the possibilities. Always check that the result and its
corresponding uncertainty have the same units. If they do not,
something went wrong.

\subsubsection*{Addition/Subtraction}

\begin{quote}
\textit{When adding or subtracting, add absolute uncertainties in
quadrature.}
\end{quote}

For example, if $d = a+b-c$, then 
\begin{equation}
\sigma_d = \sqrt{\sigma_a^2+\sigma_b^2+\sigma_c^2}
\end{equation}

\subsubsection*{Multiplication/Division}

\begin{quote}
\textit{When multiplying or dividing, add relative (fractional)
uncertainties in quadrature.}
\end{quote}

For example, if $d = \frac{ab}{c}$, then
\begin{equation}
\frac{\sigma_d}{d} = \sqrt{
\left( \frac{\sigma_a}{a} \right)^2
+\left( \frac{\sigma_b}{b} \right)^2
+\left( \frac{\sigma_c}{c} \right)^2
}
\end{equation}
When raising a value to a power, multiply its relative error by the
power. For example, if $d = \frac{a^lb^m}{c^n}$
\begin{equation}
\frac{\sigma_d}{d} = \sqrt{
\left(l \frac{\sigma_a}{a} \right)^2
+\left(m \frac{\sigma_b}{b} \right)^2
+\left(n \frac{\sigma_c}{c} \right)^2
}
\end{equation}

\subsubsection*{In General (Approximately)}

\begin{quote}
\textit{Use first derivatives to determine the approximate
  variation of the result due to the uncertainty in each measured
  quantity. Then, add them in quadrature}
\end{quote}

If a quantity $f$ is a function of the measured quantities $a, b, c,
...$, then 
\begin{equation}
\sigma_f = \sqrt{
\left(\frac{\partial f}{\partial a} \right)^2 \sigma_a^2
+\left(\frac{\partial f}{\partial b} \right)^2 \sigma_b^2
+\left(\frac{\partial f}{\partial c} \right)^2 \sigma_c^2 
+ ...
}
\end{equation}

\subsubsection*{In General (Better Approximation)}

\textit{When calculating a result which depends on measured
input quantities, determine the variations in the result due to each
input quantity, and add the variations in quadrature. In some cases,
upper and lower uncertainties differ.}

For example, if $d = a \log b$, the individual variances are 
\begin{eqnarray}
\nonumber
\sigma_{da+} &=& |(a+\sigma_a) \log b - d|\\
\nonumber
\sigma_{da-} &=& |(a-\sigma_a) \log b - d|\\
\nonumber
\sigma_{db+} &=& |a \log (b+\sigma_b) - d|\\
\sigma_{db-} &=& |a \log (b-\sigma_b) - d|
\end{eqnarray}
and the upper and lower uncertainties are
\begin{eqnarray}
\nonumber
\sigma_{d+} &=& \sqrt{\sigma_{da+}^2 + \sigma_{db+}^2}\\
\sigma_{d-} &=& \sqrt{\sigma_{da-}^2 + \sigma_{db-}^2}
\end{eqnarray}
This kind of analysis is a good job for a spreadsheet or a Python 
program. This method is approximate, because interactions between 
variables are neglected.

\subsection*{Reporting Results with Uncertainties}

\subsubsection*{Explicit Uncertainties}

Results with uncertainties are typically reported in the form
\begin{equation}
x \pm \sigma_x 
\end{equation}
Units are \textit{always} included, and are usually given after the
result and its uncertainty. It is common practice to round
uncertainties to one significant figure. Results should be rounded off
to the decimal place of the corresponding uncertainties. For example,
if an analysis of several measurements of my height reveals an average
of $\bar{h} = 1.8037$~m with a standard deviation of the mean of
$\sigma_{\bar{h}} = 0.00566$~m, I report my height as $h = 1.804 \pm
0.006$~m. The form
\begin{equation}
x(\sigma_x)
\end{equation}
is also sometimes used, where the uncertainty is given as a single
digit. In this form, my height is $h = 1.804(6)$~m. The uncertainty is
assumed to be in the last reported digit of the result. With
asymmetric uncertainties, one uses the form 
\begin{equation}
x^{+\sigma_{x+}}_{-\sigma_{x-}}
\end{equation}

Always include explicit uncertainties when reporting results.

\subsubsection*{Significant Figures}

Often, a numerical value is given without an explicit uncertainty. You
should never do this when reporting results of measurements in lab,
but we do this all the time when we are solving problems in class, for
homework, and on exams. When you write a numerical value without an
uncertainty, you imply an uncertainty with the number of
\textbf{significant figures} you include.
\begin{itemize}
\item The implied uncertainty is in the last digit on the right -- the
  \textbf{least significant digit}.
  \begin{quote}
    \textit{When I say that I am $182$ cm tall, the implied uncertainty
      is in the 2.}
  \end{quote}

\item The number of significant figures = the number of digits
  ignoring leading zeros.
  \begin{quote}
    \textit{$0.000368$ has 3 significant figures. None of the zeros
      are significant, because they are leading zeros.}
  \end{quote}

\item Notation: Where trailing zeros make the number of significant
  figures ambiguous, put a bar over the least significant figure, or
  use scientific notation. (This is also a way to indicate the least
  significant digit when keeping extra digits in intermediate values
  in multi-step calculations to avoid round-off error.)
  \begin{quote}
    \textit{$52,710$ might have has 4 or 5 significant figures. The zero
      is needed in order to express the number, either way. Let's say
      it has 4 significant figures. It can be written unambiguously
      as $52,7\bar{1}0$ or $5.271 \times 10^4$.} 
  \end{quote}

\item Addition/subtraction: Round the result to the last decimal place
  of the least precise input.
  \begin{quote}
    \textit{$5.827 + 0.61 = 6.4\bar{3}7 = 6.43$}
  \end{quote}
  
\item Multiplication/division: The result has the same number of
  significant figures as the input with the smaller number of
  significant figures.
  \begin{quote}
    \textit{$5.827 \times 0.61 = 3.\bar{5}5447 = 3.6$}
  \end{quote}
  
\end{itemize}

{\footnotesize
  \noindent
  \hrulefill
  
  \noindent
  This work is licensed under the Creative Commons
  Attribution-ShareAlike 4.0 International License: 
  \url{http://creativecommons.org/licenses/by-sa/4.0/}.\\

  \noindent
  L.A. Riley (\texttt{lriley@ursinus.edu}), updated \today
}

\end{document}
