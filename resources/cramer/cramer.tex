\documentclass[12pt]{article}

\usepackage{graphics}
\usepackage{html}
\usepackage{amssymb}
\usepackage{hyperref}

\renewcommand{\familydefault}{\sfdefault}

\setlength{\textwidth} {6.5 true in}
\setlength{\textheight}{9 true in}
\setlength{\hoffset}   {-0.50 true in}
\setlength{\voffset}   {-0.75 true in}

\begin{document}

\begin{latexonly}
\subsection*{Cramer's Rule}
\end{latexonly}

Cramer's rule is a method of solving $n$ simultaneous equations
for $n$ unknowns. 

\begin{itemize}
\item Any system of equations of this kind can be written
in the form 

\begin{equation}
\begin{array}{c}
a_{11} \, x_1 + a_{12} \, x_2 + ... + a_{1n} \, x_n = b_{1} \\
a_{21} \, x_1 + a_{22} \, x_2 + ... + a_{2n} \, x_n = b_{2} \\
\vdots \\
a_{n1} \, x_1 + a_{n2} \, x_n + ... + a_{nn} \, x_n = b_{n} 
\end{array}
\label{eq:neq}
\end{equation}

\item The coefficients on the left side of Eq.'s~1 %\ref{eq:neq}
  can be written as a matrix, 

\begin{equation}
A = 
\left(
\begin{array}{cccc}
a_{11}  & a_{12} & ...    & a_{1n} \\
a_{21}  & a_{22} & ...    & a_{2n} \\
\vdots  &\vdots  & \ddots & \vdots \\
a_{n1}  & a_{n2} & ...    & a_{nn}
\end{array}
\right)
\end{equation}

\item The value of any of the $x_i$ can be found via

\begin{equation}
x_i = \frac{|B_i|}{|A|}
\label{eq:sols}
\end{equation}

\noindent
where the notation $|A|$ and $|B_i|$ denote the 
determinants of matrices $A$ and $B_i$, and the matrix $B_i$ is
obtained by replacing the $i^{th}$ column of matrix $A$ with the
coefficients on the left side of Eq.'s~1 %\ref{eq:neq}.
For example,

\begin{equation}
B_2 = 
\left(
\begin{array}{cccc}
a_{11} & b_1    & ...    & a_{1n} \\
a_{21} & b_2    & ...    & a_{2n} \\
\vdots & \vdots & \ddots & \vdots \\
a_{n1} & b_n    & ...    & a_{nn}
\end{array}
\right)
\end{equation}

\end{itemize}

\subsubsection*{The $\mathbf{n=2}$ Case}
\begin{itemize}
\item Two equations involving two unknowns can be written in the form 

\begin{equation}
\begin{array}{c}
a_{11} \, x_1 + a_{12} \, x_2 = b_{1} \\
a_{21} \, x_1 + a_{22} \, x_2 = b_{2} 
\end{array}
\end{equation}

\noindent
which yields

\begin{equation}
A = 
\left(
\begin{array}{cc}
a_{11} & a_{12} \\
a_{21} & a_{22} 
\end{array}
\right)
\end{equation}

\item Eq.~3 %\ref{eq:sols}
  gives

\begin{equation}
x_1 = 
\frac{
\left|
\begin{array}{cc}
b_1 & a_{12} \\
b_2 & a_{22} 
\end{array}
\right|
}{
\left|
\begin{array}{cc}
a_{11} & a_{12} \\
a_{21} & a_{22} 
\end{array}
\right|
}
=
\frac{b_1 \, a_{22} - a_{12} \, b_2}
{a_{11} \, a_{22} - a_{12} \, a_{21}}
\end{equation}

\noindent
and

\begin{equation}
x_2 = \frac{
\left|
\begin{array}{cc}
a_{11} & b_1 \\
a_{21} & b_2 
\end{array}
\right|
}{
\left|
\begin{array}{cc}
a_{11} & a_{12} \\
a_{21} & a_{22} 
\end{array}
\right|
}
=
\frac{a_{11} \, b_2 - b_1 \, a_{21}}
{a_{11} \, a_{22} - a_{12} \, a_{21}}
\end{equation}

\end{itemize}

\subsubsection*{An $\mathbf{n=2}$ Example}

\begin{itemize}
\item The system of equations

\begin{equation}
\begin{array}{c}
2 \, x_1 - \, x_2 = 1 \\
5 \, x_1 + 3 \, x_2 = 2
\end{array}
\label{eq:2eq}
\end{equation}

\noindent
has solutions

\begin{equation}
x_1 = 
\frac{
\left|
\begin{array}{cc}
1 & -1 \\
2 & 3 
\end{array}
\right|
}{
\left|
\begin{array}{cc}
2 & -1 \\
5 & 3 
\end{array}
\right|
}
= \frac{5}{11}
\end{equation}

\noindent
and

\begin{equation}
x_2 = 
\frac{
\left|
\begin{array}{cc}
2 & 1 \\
5 & 2 
\end{array}
\right|
}{
\left|
\begin{array}{cc}
2 & -1 \\
5 & 3 
\end{array}
\right|
}
= -\frac{1}{11}
\end{equation}

\item The results can be verified by substituting them into either of
Eq.'s~9,%\ref{eq:2eq},

\begin{equation}
\frac{10}{11} + \frac{1}{11}  = \frac{11}{11} = 1
\hspace{24pt}\checkmark
\end{equation}

\end{itemize}

\subsubsection*{An $\mathbf{n=3}$ Example}
\begin{itemize}
\item The system of equations

\begin{equation}
\begin{array}{c}
2 \, x_1 - ~ \, x_2 + 4 \, x_3 = 2 \\
5 \, x_1 + 3 \, x_2 + 2 \, x_3 = 1 \\
~~~ \, x_1 + 6 \, x_2 + ~~ \, x_3 = -3 \\
\end{array}
\label{eq:3eq}
\end{equation}

\noindent
has solutions

\begin{equation}
\nonumber
x_1 = 
\frac{
\left|
\begin{array}{ccc}
 2 & -1 & 4 \\
 1 &  3 & 2 \\
-3 &  6 & 1
\end{array}
\right|
}{
\left|
\begin{array}{ccc}
 2 & -1 & 4 \\
 5 &  3 & 2 \\
 1 &  6 & 1
\end{array}
\right|
}
\end{equation}

\begin{equation}
x_1 = 
\frac{
    2 (3 \cdot 1 - 2 \cdot 6)
- (-1)(1 \cdot 1 - 2 \cdot -3)
+   4 (1 \cdot 6 - 3 \cdot -3)
}{
    2 (3 \cdot 1 - 2 \cdot 6)
- (-1)(5 \cdot 1 - 2 \cdot 1)
+   4 (5 \cdot 6 - 3 \cdot 1)
} = \frac{49}{93}
\end{equation}

\begin{equation}
x_2 =
\frac{
\left|
\begin{array}{ccc}
 2 &  2 & 4 \\
 5 &  1 & 2 \\
 1 & -3 & 1
\end{array}
\right|
}{
\left|
\begin{array}{ccc}
 2 & -1 & 4 \\
 5 &  3 & 2 \\
 1 &  6 & 1
\end{array}
\right|
}
\end{equation}

\begin{equation}
=  
\frac{
  2 (1 \cdot 1 - 2 \cdot -3)
- 2 (5 \cdot 1 - 2 \cdot  1)
+ 4 (5 \cdot -3 - 1 \cdot 1)
}{93} 
= -\frac{56}{93}
\end{equation}

and

\begin{equation}
x_3 =
\frac{
\left|
\begin{array}{ccc}
 2 & -1 &  2 \\
 5 &  3 &  1 \\
 1 &  6 & -3
\end{array}
\right|
}{
\left|
\begin{array}{ccc}
 2 & -1 & 4 \\
 5 &  3 & 2 \\
 1 &  6 & 1
\end{array}
\right|
}
\end{equation}

\begin{equation}
=
\frac{
    2 (3 \cdot -3 - 1 \cdot 6)
- (-1)(5 \cdot -3 - 1 \cdot 1)
+   2 (5 \cdot  6 - 3 \cdot 1)
}
{93}
=
\frac{8}{93}
\end{equation}

\item The results can be verified by substituting them into any of
Eq.'s~13,%\ref{eq:3eq}, 

\begin{equation}
\frac{2 \cdot 49 - (-56) + 4 \cdot 8}{93} = \frac{186}{93} = 2
\hspace{24pt}\checkmark
\end{equation}

\end{itemize}

{\footnotesize
  \noindent
  \hrulefill
  
  \noindent
  This work is licensed under the Creative Commons
  Attribution-ShareAlike 4.0 International License: 
  \url{http://creativecommons.org/licenses/by-sa/4.0/}.\\

  \noindent
  L.A. Riley (\texttt{lriley@ursinus.edu}), updated June 2021
}

\end{document}
